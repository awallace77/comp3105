\documentclass[12pt]{article}

\usepackage{fullpage}
\usepackage{fancybox}
\usepackage{graphicx}
\usepackage{amssymb}
\usepackage{amsmath}
\usepackage{enumerate}

%%%%%%%%%%%%%% Capsule %%%%%%%%%%%%%%%%%%%%%%%%%%%%%%%%%%%%%%%%%%%
\newcommand{\capsule}[2]{\vspace{0.5em}
  \shadowbox{%
    \begin{minipage}{.90\linewidth}%
      \textbf{#1:}~#2%
    \end{minipage}}
  \vspace{0.5em} }
%%%%%%%%%%%%%%%%%%%%%%%%%%%%%%%%%%%%%%%%%%%%%%%%%%%%%%%%%%%%%%%%%%

\newcounter{ques}
\newenvironment{question}{\stepcounter{ques}{\noindent\bf Question \arabic{ques}:}}{\vspace{5mm}}

\begin{document} 

\begin{center} \Large\bf
COMP 3105 -- Assignment 1 Report\\
-- Fall 2025 -- 
\end{center} 

\begin{center}
{\bf Due:} Sunday September 28, 2025 23:59. \\
Group 51 \\
Andrew Wallace - 101210291\\
\end{center}

\vspace{0.5em}
\newpage 

\begin{question} \textbf{(7.5\%) Linear Regression}

  \begin{enumerate}[(a)] 
    \item (1\%) $L_2$ Regression \\
    Please see A1codes.py for implementation.
    \item (3\%) $L_\infty$ Regression \\
      Here we are going to solve the $L_\infty$ loss regression problem
      \begin{center}
        $\mathbf{w} = \text{argmin}_{\mathbf{w} \in \mathbb{R}^d} = ||X \mathbf{w} - \mathbf{y}||$
      \end{center}
      Recall that this optimization can be expressed as a linear programming problem with the joint paramters $\begin{bmatrix}
        \mathbf{w} \\[-0.5em] 
        \substack{d \times 1} \\[0.15em]
        \delta \\[-0.5em] 
        \substack{1 \times 1} \\[0.15em]
      \end{bmatrix} \in \mathbb{R}^{d+1}$ as follows
      \begin{center}
        $\text{min}_{\mathbf{w}, \delta} \delta$ \\
        s.t. \\
        $\delta \ge 0 \iff -\delta \le 0$ \\
        $X \mathbf{w} - \mathbf{y} \preceq \delta \cdot \mathbf{1}_n \iff X \mathbf{w} - \delta \cdot \mathbf{1}_n \preceq \mathbf{y}$ \\
        $\mathbf{y} - X \mathbf{w} \preceq \delta \cdot \mathbf{1}_n \iff -X \mathbf{w} - \delta \cdot \mathbf{1}_n \preceq -\mathbf{y}$
      \end{center}
      In the following answers, we will convert the optimization to a form that is solvable by the \textbf{cxvopt} linear programming (LP) solver, which solves the following form of LP
      \begin{center}
        $\text{min}_{\mathbf{u}}\mathbf{c}^T \mathbf{u}$ \\
        s.t. $G \mathbf{u} \preceq \mathbf{h}$
      \end{center}
      Let the unknown variables be $\mathbf{u} = \begin{bmatrix}
        \mathbf{w} \\[-0.5em] 
        \substack{d \times 1} \\[0.15em]
        \delta \\[-0.5em] 
        \substack{1 \times 1} \\[0.15em]
      \end{bmatrix} \in \mathbb{R}^{d+1}$ \\
      For the constraints, since we have three sets of constraints, the matrix $G$ and $\mathbf{h}$ can be decomposed into three parts 
      \begin{center}
        $G \cdot \mathbf{u} = 
        \begin{bmatrix}
          G^{(1)} \\[-0.5em] 
          \substack{1 \times (d+1)} \\[0.15em]
          G^{(2)} \\[-0.5em] 
          \substack{n \times (d+1)} \\[0.15em]
          G^{(3)} \\[-0.5em] 
          \substack{n \times (d+1)} \\[0.15em]
        \end{bmatrix} \cdot 
        \begin{bmatrix}
          \mathbf{w} \\
          \delta 
        \end{bmatrix} \preceq 
        \begin{bmatrix}
          \mathbf{h}^{(1)} \\[-0.5em] 
          \substack{1 \times 1} \\[0.15em]
          \mathbf{h}^{(2)} \\[-0.5em] 
          \substack{d \times 1} \\[0.15em]
          \mathbf{h}^{(3)} \\[-0.5em] 
          \substack{d \times 1} \\[0.15em]
        \end{bmatrix}$
      \end{center}
    \begin{enumerate}[(b.1)]
      \item (0.25\%) For the objective function, we want $\mathbf{c}^T \mathbf{u} = \delta$. What should $\mathbf{c} \in \mathbb{R}^{d+1}$ be? 
      Recall that $\mathbf{u} = \begin{bmatrix} \mathbf{w} \\ \delta \end{bmatrix}$ \\
      For the objective function to be $\delta$ we have: \\
      $\mathbf{c}^T \mathbf{u} = [c_1, c_2, \ldots c_d, c_{d+1}] \cdot 
      \begin{bmatrix}
        w_1 \\
        w_2 \\
        \vdots \\
        w_d \\
        \delta 
        \end{bmatrix} = c_1 w_1 + c_2 w_2 + \ldots + w_d u_d + c_{d+1} \delta$ \\
      Now let $c_1 = c_2 = \ldots = c_d = 0$ \\
      And $c_{d+1} = 1$ \\
      This gives us: \\
      $0w_1 + 0w_2 + \ldots 0w_d + \delta$ \\
      $=\delta$. \\
      Thus $\mathbf{c} = \begin{bmatrix}
        0 \\
        0 \\
        \vdots \\
        0 \\
        1
        \end{bmatrix}$ \\
      Where $c_1 = c_2 = \ldots = c_d =0$ and $c_{d+1} = 1$.
      \item (0.25\%) We want $G^{(1)}\mathbf{u} \preceq \mathbf{h}^{(1)} \iff \delta \ge 0.$ What should $G^{(1)} \in \mathbb{R}^{1 \times (d+1)}$ and $h^{(1)} \in \mathbb{R}$ be? \\
      Recall the first constraint: $-\delta \le 0$ \\
      We want
      \begin{center}
        $\underset{1 \times (d + 1)}{G^{(1)}} \cdot \mathbf{u} \preceq \underset{1 \times 1}{\mathbf{h}^{(1)}}$ \\
      \end{center}
      We will now map our constraint into this form.
      \begin{center}
        $\begin{bmatrix}
          \underset{1 \times d}{G^{(11)}} && \underset{1 \times 1}{G^{(12)}} 
        \end{bmatrix} \cdot 
        \begin{bmatrix}
          \mathbf{w} \\
          \delta
        \end{bmatrix} \preceq 
        \underset{n \times 1}{\mathbf{h}^{(1)}} \iff -\delta \le 0$ \\
        $\underset{1 \times d}{G^{(11)}} \cdot \mathbf{w} + \underset{1 \times 1}{G^{(12)}} \cdot \delta = -\delta \le 0 = \underset{1 \times 1}{\mathbf{h}^{(1)}}$ \\
      \end{center}
      So, \\
      $\underset{1 \times d}{G^{(11)}} = \mathbf{0}_{1 \times d}$ \\
      $\underset{1 \times 1}{G^{(12)}} = -1$ \\
      $\underset{1 \times (d+1)}{G^{(1)}} = 
      \begin{bmatrix}
        \mathbf{0}_{1 \times d} & -1 
      \end{bmatrix}$ \\
      $\underset{1 \times 1}{\mathbf{h}^{(1)}} = \mathbf{0}_{1 \times 1}$
      \item (0.25\%) We want $G^{(2)}\mathbf{u} \preceq \mathbf{h}^{(2)} \iff X \mathbf{w} - \mathbf{y} \preceq \delta \cdot \mathbf{1_n}.$ What should $G^{(2)} \in \mathbb{R}^{n \times (d+1)}$ and $h^{(2)} \in \mathbb{R}^n$ be? \\
      Recall our second constraint is \\
      $X \mathbf{w} - \mathbf{y} \preceq \delta \cdot \mathbf{1}_n \iff X \mathbf{w} - \delta \cdot \mathbf{1}_n \preceq \mathbf{y}$ \\
      We want
      \begin{center}
        $\underset{n \times (d + 1)}{G^{(2)}} \cdot \mathbf{u} \preceq \underset{n \times 1}{\mathbf{h}^{(2)}}$ \\
      \end{center}
      We will now map our constraint into this form.
      \begin{center}
        $\begin{bmatrix}
          \underset{n \times d}{G^{(21)}} && \underset{n \times 1}{G^{(22)}} 
        \end{bmatrix} \cdot 
        \begin{bmatrix}
          \mathbf{w} \\
          \delta
        \end{bmatrix} \preceq 
        \underset{n \times 1}{\mathbf{h}^{(2)}} \iff X \mathbf{w} - \delta \cdot \mathbf{1}_n \preceq \mathbf{y}$ \\
        $\underset{n \times d}{G^{(21)}} \cdot \mathbf{w} + \underset{n \times 1}{G^{(22)}} \cdot \delta = X \mathbf{w} - \delta \cdot \mathbf{1}_n \preceq \mathbf{y} = \underset{n \times 1}{\mathbf{h}^{(1)}}$ \\
      \end{center}
      So, \\
      $\underset{n \times d}{G^{(21)}} = X$ \\
      $\underset{n \times 1}{G^{(22)}} = -\mathbf{1}_{n \times 1}$ \\
      $\underset{n \times (d+1)}{G^{(2)}} = 
      \begin{bmatrix}
        X & -\mathbf{1}_{n \times 1}
      \end{bmatrix}$ \\
      $\underset{n \times 1}{h^{(2)}} = \mathbf{y}$
      \item (0.25\%) We want $G^{(3)}\mathbf{u} \preceq \mathbf{h}^{(3)} \iff \mathbf{y} - X \mathbf{w} \preceq \delta \cdot \mathbf{1_n}.$ What should $G^{(3)} \in \mathbb{R}^{n \times (d+1)}$ and $h^{(3)} \in \mathbb{R}^n$ be?\\
      Recall our third constraint is \\
      $\mathbf{y} - X \mathbf{w} \preceq \delta \cdot \mathbf{1}_n \iff -X \mathbf{w} - \delta \cdot \mathbf{1}_n \preceq -\mathbf{y}$ \\
      We want
      \begin{center}
        $\underset{n \times (d + 1)}{G^{(3)}} \cdot \mathbf{u} \preceq \underset{n \times 1}{\mathbf{h}^{(3)}}$ \\
      \end{center}
      We will now map our constraint into this form.
      \begin{center}
        $\begin{bmatrix}
          \underset{n \times d}{G^{(31)}} && \underset{n \times 1}{G^{(32)}} 
        \end{bmatrix} \cdot 
        \begin{bmatrix}
          \mathbf{w} \\
          \delta
        \end{bmatrix} \preceq 
        \underset{n \times 1}{\mathbf{h}^{(3)}} \iff -X \mathbf{w} - \delta \cdot \mathbf{1}_n \preceq -\mathbf{y}$ \\
        $\underset{n \times d}{G^{(31)}} \cdot \mathbf{w} + \underset{n \times 1}{G^{(32)}} \cdot \delta = -X \mathbf{w} - \delta \cdot \mathbf{1}_n \preceq -\mathbf{y} = \underset{n \times 1}{\mathbf{h}^{(3)}}$ \\
      \end{center}
      So, \\
      $\underset{n \times d}{G^{(31)}} = -X$ \\
      $\underset{n \times 1}{G^{(32)}} = -\mathbf{1}_{n \times 1}$ \\
      $\underset{n \times (d+1)}{G^{(3)}} = 
      \begin{bmatrix}
        -X & -\mathbf{1}_{n \times 1}
      \end{bmatrix}$ \\
      $\underset{n \times 1}{h^{(3)}} = -\mathbf{y}$
      
      \item (2\%) Based on your derivations in (b), implement a Python function \begin{center}$\mathbf{w} = \text{minimizeLinf}(X,y)$ \end{center} that returns a $d \times 1$ vector of weights/parameters $\mathbf{w}$ corresponding to the solution of minimum $L_\infty$ loss.
      \begin{center}
        $\mathbf{w} = \text{argmin}_{w \in \mathbb{R}^d} || X \mathbf{w} - \mathbf{y}||$
      \end{center}

      Based on our derivations in (b), we result in \\
      $\mathbf{c} = \begin{bmatrix}
        0 \\
        0 \\ 
        \vdots \\
        1
      \end{bmatrix} \in \mathbb{R}^{d+1}$ \\[0.5em]
      $\mathbf{u} = \begin{bmatrix}
        \mathbf{w} \\
        \delta
      \end{bmatrix} \in \mathbb{R}^{d+1}$ \\[0.5em]
      $G = \begin{bmatrix}
        \mathbf{0}_{1 \times d} & -1 \\
        X & -\mathbf{1}_n \\
        -X & -\mathbf{1}_n
      \end{bmatrix} \in \mathbb{R}^{(2n+1) \times (d+1)}$ \\[0.5em]
      $\mathbf{h} = \begin{bmatrix}
        0 \\
        \mathbf{y} \\
        -\mathbf{y}
      \end{bmatrix} \in \mathbb{R}^{2n+1}$ 
      \begin{center}
        s.t. \\
        $\underset{(2n+1) \times (d+1)}{G} \cdot \underset{(d+1) \times 1}{\mathbf{u}} \preceq \underset{(2n+1) \times 1}{\mathbf{h}}$
      \end{center}


      Please see A1codes.py for full implementation. 
    \end{enumerate}

    \item (2\%) Synthetic Regression Problem  \\
    In this part, you will evaluate your implemented algorithms on a synthetic dataset.
    \begin{enumerate}[(c.1)]
      \item (1\%) Implement a Python function 
      \begin{verbatim}
        train loss, test loss = synRegExperiments() 
      \end{verbatim} that returns a 2 x 2 matrix train loss of average training losses and a 2 x 2 matrix test loss of average test losses (See Table 1 and Table 2 below.) It repeats 100 runs as follows \\
      Please see \textbf{A1codes.py} for full details.
      \item (1\%) Looking at your tables from above, analyze the results and discuss any findings you may have and the possible reason behind them.
    \end{enumerate}
  \end{enumerate}

\end{question}

\end{document} 
